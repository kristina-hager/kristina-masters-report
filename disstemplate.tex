%%%%%%%%%%%%%%%%%%%%%%%%%%%%%%%%%%%%%%%%%%%%%%%%%%%%%%%%%%%%%%%%%%%%%%
%%  disstemplate.tex, to be compiled with latex.		     %
%%  08 April 2002	Version 4				     %
%%%%%%%%%%%%%%%%%%%%%%%%%%%%%%%%%%%%%%%%%%%%%%%%%%%%%%%%%%%%%%%%%%%%%%
%%								     %
%%  Writing a Doctoral Dissertation with LaTeX at		     %
%%	the University of Texas at Austin			     %
%%								     %
%%  (Modify this ``template'' for your own dissertation.)	     %
%%								     %
%%%%%%%%%%%%%%%%%%%%%%%%%%%%%%%%%%%%%%%%%%%%%%%%%%%%%%%%%%%%%%%%%%%%%%


\documentclass[12pt]{report}	% The documentclass must be ``report''.

\usepackage{utdiss2}  		% Dissertation package style file.


%%%%%%%%%%%%%%%%%%%%%%%%%%%%%%%%%%%%%%%%%%%%%%%%%%%%%%%%%%%%%%%%%%%%%%
% Optional packages used for this sample dissertation. If you don't  %
% need a capability in your dissertation, feel free to comment out   %
% the package usage command.					     %
%%%%%%%%%%%%%%%%%%%%%%%%%%%%%%%%%%%%%%%%%%%%%%%%%%%%%%%%%%%%%%%%%%%%%%

\usepackage{amsmath,amsthm,amsfonts,amscd} 
				% Some packages to write mathematics.
\usepackage{eucal} 	 	% Euler fonts
\usepackage{verbatim}      	% Allows quoting source with commands.
\usepackage{makeidx}       	% Package to make an index.
\usepackage{graphicx}         	% Allows inclusion of image (eps, etc) files
\usepackage{epsfig}         	% Allows inclusion of eps files.
\usepackage{citesort}         	% 
\usepackage{url}		% Allows good typesetting of web URLs.
\usepackage{draftcopy}		% Uncomment this line to have the
				% word, "DRAFT," as a background
				% "watermark" on all of the pages of
				% of your draft versions. When ready
				% to generate your final copy, re-comment
				% it out with a percent sign to remove
				% the word draft before you re-run
				% Makediss for the last time.

\author{Kristina Denise Hager}  	% Required

\address{4806 Savorey Lane\\ Austin, Texas 78744}  % Required

\title{Patterns for Modular Android Development}
                                                    % Required

%%%%%%%%%%%%%%%%%%%%%%%%%%%%%%%%%%%%%%%%%%%%%%%%%%%%%%%%%%%%%%%%%%%%%%
% NOTICE: The total number of supervisors and other members %%%%%%%%%%
%%%%%%%%%%%%%%% MUST be seven (7) or less! If you put in more, %%%%%%%
%%%%%%%%%%%%%%% they are put on the page after the Committee %%%%%%%%%
%%%%%%%%%%%%%%% Certification of Approved Version page. %%%%%%%%%%%%%%
%%%%%%%%%%%%%%%%%%%%%%%%%%%%%%%%%%%%%%%%%%%%%%%%%%%%%%%%%%%%%%%%%%%%%%

%%%%%%%%%%%%%%%%%%%%%%%%%%%%%%%%%%%%%%%%%%%%%%%%%%%%%%%%%%%%%%%%%%%%%%
%
% Enter names of the supervisor and co-supervisor(s), if any,
% of your dissertation committee. Put one name per line with
% the name in square brackets. The name on the last line, however,
% must be in curly braces.
%
% If you have only one supervisor, the entry below will read:
%
%	\supervisor
%		{Supervisor's Name}
%
% NOTE: Maximum three supervisors. Minimum one supervisor.
% NOTE: The Office of Graduate Studies will accept only two supervisors!
% 
%
\supervisor
	{Adnan Aziz}

%%%%%%%%%%%%%%%%%%%%%%%%%%%%%%%%%%%%%%%%%%%%%%%%%%%%%%%%%%%%%%%%%%%%%%
%
% Enter names of the other (non-supervisor) members(s) of your
% dissertation committee. Put one name per line with the name
% in square brackets. The name on the last line, however, must
% be in curly braces.
%
% NOTE: Maximum six other members. Minimum zero other members.
% NOTE: The Office of Graduate Studies may restrict you to a total
%	of six committee members.
%
%
\committeemembers
	{Christine Julien}

%%%%%%%%%%%%%%%%%%%%%%%%%%%%%%%%%%%%%%%%%%%%%%%%%%%%%%%%%%%%%%%%%%%%%%

\previousdegrees{B.S.}
     % The abbreviated form of your previous degree(s).
     % E.g., \previousdegrees{B.S., MBA}.
     %
     % The default value is `B.S., M.S.'

%\graduationmonth{...}      
     % Graduation month, either May, August, or December, in the form
     % as `\graduationmonth{May}'. Do not abbreviate.
     %
     % The default value (either May, August, or December) is guessed
     % according to the time of running LaTeX.

%\graduationyear{...}   
     % Graduation year, in the form as `\graduationyear{2001}'.
     % Use a 4 digit (not a 2 digit) number.
     %
     % The default value is guessed according to the time of 
     % running LaTeX.

%\typist{...}       
     % The name(s) of typist(s), put `the author' if you do it yourself.
     % E.g., `\typist{Maryann Hersey and the author}'.
     %
     % The default value is `the author'.


%%%%%%%%%%%%%%%%%%%%%%%%%%%%%%%%%%%%%%%%%%%%%%%%%%%%%%%%%%%%%%%%%%%%%%
% Commands for master's theses and reports.			     %
%%%%%%%%%%%%%%%%%%%%%%%%%%%%%%%%%%%%%%%%%%%%%%%%%%%%%%%%%%%%%%%%%%%%%%
%
% If the degree you're seeking is NOT Doctor of Philosophy, uncomment
% (remove the % in front of) the following two command lines (the ones
% that have the \ as their second character).
%
\degree{MASTER OF SCIENCE IN ENGINEERING}
\degreeabbr{M.S.E.}

% Uncomment the line below that corresponds to the type of master's
% document you are writing.
%
\masterreport
%\masterthesis


%%%%%%%%%%%%%%%%%%%%%%%%%%%%%%%%%%%%%%%%%%%%%%%%%%%%%%%%%%%%%%%%%%%%%%
% Some optional commands to change the document's defaults.	     %
%%%%%%%%%%%%%%%%%%%%%%%%%%%%%%%%%%%%%%%%%%%%%%%%%%%%%%%%%%%%%%%%%%%%%%
%
%\singlespacing
%\oneandonehalfspacing

%\singlespacequote
\oneandonehalfspacequote

\topmargin 0.125in	% Adjust this value if the PostScript file output
			% of your dissertation has incorrect top and 
			% bottom margins. Print a copy of at least one
			% full page of your dissertation (not the first
			% page of a chapter) and measure the top and
			% bottom margins with a ruler. You must have
			% a top margin of 1.5" and a bottom margin of
			% at least 1.25". The page numbers must be at
			% least 1.00" from the bottom of the page.
			% If the margins are not correct, adjust this
			% value accordingly and re-compile and print again.
			%
			% The default value is 0.125"

		% If you want to adjust other margins, they are in the
		% utdiss2-nn.sty file near the top. If you are using
		% the shell script Makediss on a Unix/Linux system, make
		% your changes in the utdiss2-nn.sty file instead of
		% utdiss2.sty because Makediss will overwrite any changes
		% made to utdiss2.sty.

%%%%%%%%%%%%%%%%%%%%%%%%%%%%%%%%%%%%%%%%%%%%%%%%%%%%%%%%%%%%%%%%%%%%%%
% Some optional commands to be tested.				     %
%%%%%%%%%%%%%%%%%%%%%%%%%%%%%%%%%%%%%%%%%%%%%%%%%%%%%%%%%%%%%%%%%%%%%%

% If there are 10 or more sections, 10 or more subsections for a section,
% etc., you need to make an adjustment to the Table of Contents with the
% command \longtocentry.
%
%\longtocentry 



%%%%%%%%%%%%%%%%%%%%%%%%%%%%%%%%%%%%%%%%%%%%%%%%%%%%%%%%%%%%%%%%%%%%%%
%	Some math support.					     %
%%%%%%%%%%%%%%%%%%%%%%%%%%%%%%%%%%%%%%%%%%%%%%%%%%%%%%%%%%%%%%%%%%%%%%
%
%	Theorem environments (these need the amsthm package)
%
%% \theoremstyle{plain} %% This is the default

\newtheorem{thm}{Theorem}[section]
\newtheorem{cor}[thm]{Corollary}
\newtheorem{lem}[thm]{Lemma}
\newtheorem{prop}[thm]{Proposition}
\newtheorem{ax}{Axiom}

\theoremstyle{definition}
\newtheorem{defn}{Definition}[section]

\theoremstyle{remark}
\newtheorem{rem}{Remark}[section]
\newtheorem*{notation}{Notation}

%\numberwithin{equation}{section}


%%%%%%%%%%%%%%%%%%%%%%%%%%%%%%%%%%%%%%%%%%%%%%%%%%%%%%%%%%%%%%%%%%%%%%
%	Macros.							     %
%%%%%%%%%%%%%%%%%%%%%%%%%%%%%%%%%%%%%%%%%%%%%%%%%%%%%%%%%%%%%%%%%%%%%%
%
%	Here some macros that are needed in this document:


\newcommand{\latexe}{{\LaTeX\kern.125em2%
                      \lower.5ex\hbox{$\varepsilon$}}}

\newcommand{\amslatex}{\AmS-\LaTeX{}}

\chardef\bslash=`\\	% \bslash makes a backslash (in tt fonts)
			%	p. 424, TeXbook

\newcommand{\cn}[1]{\texttt{\bslash #1}}

\makeatletter		% Starts section where @ is considered a letter
			% and thus may be used in commands.
\def\square{\RIfM@\bgroup\else$\bgroup\aftergroup$\fi
  \vcenter{\hrule\hbox{\vrule\@height.6em\kern.6em\vrule}%
                                              \hrule}\egroup}
\makeatother		% Ends sections where @ is considered a letter.
			% Now @ cannot be used in commands.

\makeindex    % Make the index

%%%%%%%%%%%%%%%%%%%%%%%%%%%%%%%%%%%%%%%%%%%%%%%%%%%%%%%%%%%%%%%%%%%%%%
%		The document starts here.			     %
%%%%%%%%%%%%%%%%%%%%%%%%%%%%%%%%%%%%%%%%%%%%%%%%%%%%%%%%%%%%%%%%%%%%%%

\begin{document}

\copyrightpage          % Produces the copyright page.

%
% NOTE: In a doctoral dissertation, the Committee Certification page
%		(with signatures) is BEFORE the Title page.
%	In a masters thesis or report, the Signature page
%		(with signatures) is AFTER the Title page.
%
%	If you are writing a masters thesis or report, you MUST REVERSE
%	the order of the \commcertpage and \titlepage commands below.
%
\commcertpage           % Produces the Committee Certification
			%   of Approved Version page (doctoral)
			%   or Signature page (masters).
			%		20 Mar 2002	cwm

\titlepage              % Produces the title page.




%%%%%%%%%%%%%%%%%%%%%%%%%%%%%%%%%%%%%%%%%%%%%%%%%%%%%%%%%%%%%%%%%%%%%%
% Dedication and/or epigraph are optional, but must occur here.      %
%%%%%%%%%%%%%%%%%%%%%%%%%%%%%%%%%%%%%%%%%%%%%%%%%%%%%%%%%%%%%%%%%%%%%%
%
\begin{dedication}
\index{Dedication@\emph{Dedication}}%
This report is dedicated to my wife Jean Krejca for her encouragement,\linebreak
and to my friends and coworkers for their patience, 
and to my parents for their support.
\end{dedication}


\begin{acknowledgments}		% Optional
\index{Acknowledgments@\emph{Acknowledgments}}%
I would like to thank my supervisor, Dr. Adnan Aziz, 
for his enthusiasm and wisdom, his invaluable feedback, and for teaching one of best 
and most challenging classes I've ever taken.
In addition I would like to thank Dr. Christine Julien, Mr. Bill Bard, Dr. Kathleen Barber, and all of the other excellent instructors in the CLEE program.
\end{acknowledgments}


% The abstract is required. Note the use of ``utabstract'' instead of
% ``abstract''! This was necessary to fix a page numbering problem.
% The abstract heading is generated automatically.
% Do NOT use \begin{abstract} ... \end{abstract}.
%
\utabstract
\index{Abstract}%
\indent
This report is a case study of TODO.

\tableofcontents   % Table of Contents will be automatically
                   % generated and placed here.

\listoftables      % List of Tables and List of Figures will be placed
\listoffigures     % here, if applicable.
%\lstlistoflistings	%this could be cool to have later on


%%%%%%%%%%%%%%%%%%%%%%%%%%%%%%%%%%%%%%%%%%%%%%%%%%%%%%%%%%%%%%%%%%%%%%
% Actual text starts here.					     %
%%%%%%%%%%%%%%%%%%%%%%%%%%%%%%%%%%%%%%%%%%%%%%%%%%%%%%%%%%%%%%%%%%%%%%
%
% Including external files for each chapter makes this document simpler,
% makes each chapter simpler, and allows for generating test documents
% with as few as zero chapters (by commenting out the include statements).
% This allows quicker processing by the Makediss command file in case you
% are not working on a specific, long and slow to compile chapter. You
% can even change the chapter order by merely interchanging the order
% of the include statements (something I found helpful in my own
% dissertation).
%

\chapter{Introduction}
\index{Introduction@\emph{Introduction}}%
\label{ch:intro}

%\section{blah}
%\index{blah@\emph{blah}}%
%
%\texttt{asdf}
%
%\begin{quote}
%\index{guarantee}%
%This template package is provided and licensed ``as is'' without warranty
%of any kind, either expressed or implied, including, but not limited to,
%the implied warranties of merchantability and fitness for a particular
%purpose. Yadda, yadda, yadda, \ldots
%\end{quote}

\section{Mobile Development is Hot}
This is a sentence talking about the hotness of mobile dev.

\section{Open Source Development Trends}
\index{Open Source}
This is sentence.

\section{Code Reuse is Critical to Modern Software Development}
This is a sentence.

\section{Software Concepts Discussed in this Paper}
This is a sentence.
Cohesion\index{cohesion} is a software design concept.
Coupling\index{coupling} is also amazing.

\section{Structure of This Report}
This is a sentence.

\section{Source Code and Demonstration Resources}
This is a sentence.



\chapter{Background of Android Development}
\index{Background of Android Development@\emph{Background of Android Development}}%
\label{ch:background}

\section{The Android Development Environment}
\subsection{Integrated Development Environment (IDE)}
This is a sentence.

\subsection{Build System}
This is a sentence.

\subsection{Android Native Code}
This is a sentence.

\subsection{Command Line Tools}
This is a sentence.

\section{Introduction to Android Terms}
\subsection{Java}
This is sentence.

\subsection{The Manifest}
This is sentence.

\subsection{Resources}
This is sentence.

\subsection{Notable Components}
This is sentence.
Activity.
Service.
Intent.

\subsection{Artifacts}
This is sentence.
JAR.
AAR.
APK.

\section{Publishing Android Libraries}
\subsection{MavenCentral}
This is sentence.

\subsection{JCenter}
This is sentence.

\subsection{Private Options}
This is sentence.



\chapter{Motivation For This Work}
\index{Motivation For This Work@\emph{Motivation For This Work}}%
\label{ch:motivation}

\section{Case Study: Spatialite}
This is a sentence.

\section{Case Study: Geopaparazzi}
This is sentence.
This is another sentence~\cite{knuth:tb}.




\chapter{Overview Of Android Libraries}
\label{ch:overview}

\section{Requirements for Libraries}

\subsection{Well Defined Interfaces}
This is a sentence.
This is another sentence.

\subsection{Testing}
This is a sentence.

\subsection{Publishing}
This is sentence.

\subsection{Documentation}
This is sentence.

\subsection{Sample Applications}
This is sentence.

\section{Challenges in Creating Android Libraries}
These challenges lead the developer to missing separation of concerns~\cite{knuth:tb}.

\subsection{IDE Obstacles}
This is a sentence.

\subsection{Challenges around Automatic Archive Publishing}
This is a sentence.

\subsection{Lack of Documentation Standards}
This is a sentence.

\section{Example Android Modules}
%Remember to differentiate this section with Chapater 5

\subsection{Traditional JAR}
This is a sentence.

\subsection{The New AAR Standard}
This is a sentence.

\subsection{Comparison of JAR and AAR}
This is a sentence.


\chapter{My Android Library Best Practices}
\index{My Android Library Best Practices@\emph{My Android Library Best Practices}}
\label{ch:bestpractices}

Lorem ipsum dolor sit amet, consectetur adipiscing elit. Donec fringilla maximus rhoncus.
Morbi et purus eu magna venenatis fermentum scelerisque a est.
Pellentesque velit magna, rutrum quis porttitor id, tincidunt eu tellus.
Aenean eu posuere magna.
Etiam dignissim ullamcorper turpis, ac sollicitudin elit.
Etiam pulvinar nibh eu nisi sagittis dictum.
Morbi faucibus purus sit amet felis tempor, id luctus libero lobortis.
Morbi vitae dui sed odio vulputate tristique.
Aenean quis ipsum leo.
Sed sodales sem non quam finibus ultrices.
Aliquam erat volutpat.
Sed vel tincidunt justo.

\section{Creating Plain Java JAR Libraries}

Lorem ipsum dolor sit amet, consectetur adipiscing elit. Donec fringilla maximus rhoncus.
Morbi et purus eu magna venenatis fermentum scelerisque a est.
Pellentesque velit magna, rutrum quis porttitor id, tincidunt eu tellus.
Aenean eu posuere magna.
Etiam dignissim ullamcorper turpis, ac sollicitudin elit.

\subsection{Hello Planet: No Dependencies}
Etiam pulvinar nibh eu nisi sagittis dictum.
Morbi faucibus purus sit amet felis tempor, id luctus libero lobortis.
Morbi vitae dui sed odio vulputate tristique.
Aenean quis ipsum leo.
Sed sodales sem non quam finibus ultrices.
Aliquam erat volutpat.
Sed vel tincidunt justo.
Build.
Testing.
Publishing.

\subsection{Planetary System: Handling Dependencies}
This is a sentence.
Build.
Testing.
Publishing.

\section{Creating Android AAR Libraries}
This is a sentence.

\subsection{Creating an Intent-Based Library}
This is sentence.
Build.
Testing.
Publishing.

\subsection{Creating an Custom View Library}
This is sentence.
Build.
Testing.
Publishing.

\section{Using an Android Library}
This is a sentence.

\subsection{Using the Intent-Based Library}
This is a sentence.

\subsection{Using the Custom View Library: Entirely from XML}
This is a sentence.


\subsection{Using the Custom View Library: Entirely from Java}
This is a sentence.



\chapter{Conclusions}
\index{Conclusions@\emph{Conclusions}}%
\label{ch:conclusions}

\section{Summary of my Contributions}
This is a sentence.

\section{Best Practices and Lessons Learned}
This is sentence.

\section{Future Work}
This is a sentence.





%%%%%%%%%%%%%%%%%%%%%%%%%%%%%%%%%%%%%%%%%%%%%%%%%%%%%%%%%%%%%%%%%%%%%%
% Appendix/Appendices                                                %
%%%%%%%%%%%%%%%%%%%%%%%%%%%%%%%%%%%%%%%%%%%%%%%%%%%%%%%%%%%%%%%%%%%%%%
%
% If you have only one appendix, use the command \appendix instead
% of \appendices.
%
\appendices
\index{Appendices@\emph{Appendices}}%

% \include{chapter-appendix1}

%%%%%%%%%%%%%%%%%%%%%%%%%%%%%%%%%%%%%%%%%%
\chapter{Open Source Credits}
\index{Appendix!Open Source credits@\emph{Open Source credits}}%
\label{appendix:credits}

This work would not have been possible without the following open source components.
The components listed here are the direct dependencies.
Some of these may install other dependencies in turn.

\begin{itemize}
\item Foobar\index{Foobar} library~\cite{knuth:tb}
\item Bazbum\index{Bazbum} library
\end{itemize}


% \include{chapter-appendix3}

\printindex{}

%%%%%%%%%%%%%%%%%%%%%%%%%%%%%%%%%%%%%%%%%%%%%%%%%%%%%%%%%%%%%%%%%%%%%%
% Generate the bibliography.					     %
%%%%%%%%%%%%%%%%%%%%%%%%%%%%%%%%%%%%%%%%%%%%%%%%%%%%%%%%%%%%%%%%%%%%%%
%								     %
% NOTE: For master's theses and reports, NOTHING is permitted to     %
%	come between the bibliography and the vita. The command      %
%	to generate the index (if used) MUST be moved to before      %
%	this section.						     %
%								     %
\nocite{*}      % This command causes all items in the 		     %
                % bibliographic database to be added to 	     %
                % the bibliography, even if they are not 	     %
                % explicitly cited in the text. 		     %
		%						     %
\bibliographystyle{plain}  % Here the bibliography 		     %
\bibliography{diss}        % is inserted.			     %
\index{Bibliography@\emph{Bibliography}}%			     %
%%%%%%%%%%%%%%%%%%%%%%%%%%%%%%%%%%%%%%%%%%%%%%%%%%%%%%%%%%%%%%%%%%%%%%


%%%%%%%%%%%%%%%%%%%%%%%%%%%%%%%%%%%%%%%%%%%%%%%%%%%%%%%%%%%%%%%%%%%%%%
% Generate the index.						     %
%%%%%%%%%%%%%%%%%%%%%%%%%%%%%%%%%%%%%%%%%%%%%%%%%%%%%%%%%%%%%%%%%%%%%%
%								     %
% NOTE: For master's theses and reports, NOTHING is permitted to     %
%	come between the bibliography and the vita. This section     %
%	to generate the index (if used) MUST be moved to before      %
%	the bibliography section.				     %
%								     %
% \printindex%    % Include the index here. Comment out this line      %
%		% with a percent sign if you do not want an index.   %
%%%%%%%%%%%%%%%%%%%%%%%%%%%%%%%%%%%%%%%%%%%%%%%%%%%%%%%%%%%%%%%%%%%%%%


%%%%%%%%%%%%%%%%%%%%%%%%%%%%%%%%%%%%%%%%%%%%%%%%%%%%%%%%%%%%%%%%%%%%%%
% Vita page.							     %
%%%%%%%%%%%%%%%%%%%%%%%%%%%%%%%%%%%%%%%%%%%%%%%%%%%%%%%%%%%%%%%%%%%%%%

\begin{vita}
Kristina Hager was born in Denton, Texas on 12 April 1979, the daugher of
Kenneth G Hager and Deborah Williams Caraway.  She received the Bachelor
of Arts degree in Computer Science from the University of Texas at Austin.
She works as a software engineer in Austin, Texas and began graduate studies in 
Software Engineering at the University of Texas at Austin in January 2013.

\end{vita}

\end{document}
